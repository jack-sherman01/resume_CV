%%
%% Copyright (c) 2018-2019 Weitian LI <wt@liwt.net>
%% CC BY 4.0 License
%%
%% Résumé
%% ------
%% A short document (1-2 pages) to sum up the job-related accomplishments
%% and experience.
%%
%% Checklist
%% ---------
%% * Contact Information
%% * Work History / Experience
%% * Education
%% * Skills
%% * Summary & Objective (optional)
%% * Hobbies & Interests (optional)
%%
%% Credits
%% -------
%% * CV vs. Resume: What is the Difference? When to Use Which?
%%   https://uptowork.com/blog/cv-vs-resume-difference
%% * How to Make a Resume: A Step-by-Step Guide (+30 Examples)
%%   https://uptowork.com/blog/how-to-make-a-resume
%% * Entry-Level Resume: Sample and Complete Guide (+20 Examples)
%%   https://uptowork.com/blog/entry-level-resume-example
%%
%% Created: 2018-04-14
%%

% English version
\documentclass{resume}

% Adjust icon size (default: same size as the text)
\iconsize{\Large}

% File information shown at the footer of the last page
\fileinfo{%
  \faCopyright{} 2019--2021, Heng Zhang \hspace{0.5em}
  \creativecommons{by}{4.0} \hspace{0.5em}
  \githublink{https://github.com/jack-sherman01/}{resume} \hspace{0.5em}
  \faEdit{} \today
}

\name{Heng}{Zhang}
%\keywords{BSD, Linux, Programming, Python, C, Shell, DevOps, SysAdmin}
% \tagline{\icon{\faBinoculars}} <position-to-look-for>}
% \tagline{<current-position>}
% \photo{<height>}{<filename>}
\profile{
  \mobile{191-456-99094}
  \email{hengzhang001@tongji.edu.cn}
  \github{jack-sherman01} \\
  \degree{M.E. in Robotics}
  \university{Tongji University (TJU)}
%  \birthday{1991 Sept.}
  \address{Shanghai}
  \home{\href{https://jack-sherman01.github.io/heng.github.io/}{Heng's Page}}\\
  % Custom information:
  % \icontext{<icon>}{<text>}
  % \iconlink{<icon>}{<link>}{<text>}
}

\begin{document}
\makeheader

%======================================================================
\sectionTitle{About Me}{\faAngellist}
%======================================================================
I am now a 2nd-year master student in \href{https://rail.tongji.edu.cn/}{RAIL} Lab at Tongji University (TJU), working with Prof.\href{https://see.tongji.edu.cn/info/1233/7519.htm} {Qijun Chen}. Highly-motivated in robotics with good foundations of math and programming. Proficient in data modeling and analysis, and enthusiastic about computer and the corresponding interdisciplinary subject.

I'm interested in Robot Learning, Intelligent Motion Control, Cognitive Learning and their applications towards Artificial General Intelligence (AGI). My current research interest focuses include:
\begin{itemize}
  \item Human-Robot Interaction (collaborative robot motion control, and drag teaching, force control);
  \item Imitation learning, robot learning and representation learning;
  \item visual reasoning, attention and saliency (cognitive learning, commonsense reasoning).\\
\end{itemize}

%======================================================================
\sectionTitle{Education}{\faGraduationCap}
%======================================================================
\begin{educations}
  \education%
    {September 2019}%
    [present]%
    {Tongji University (TJU)}%
    {College of Electronic and Information Engineering}%
    {Control Engineering   \quad \textbf{GPA}:4.45}%
    {M.E.}\\

  \separator{0.5ex}
  \education%
    {September 2012}%
    [June 2016]%
    {Northeast Electric Power University}%
    {Department of Automation and Engineering}%
    {Automation\quad{\textbf{Honor}: Outstanding Undergraduate Thesis}}%
    {Bachelor's Degree}
    
\end{educations}


%======================================================================
\sectionTitle{Selected Publications}{\faGoogle}
%======================================================================
\begin{itemize}
\item \textbf{A Survey on Imitation Learning for Robot Manipulation} \quad [under review]\\
  \footnotesize{\textbf{Heng Zhang}, Xianyou Zhong*, Zhengang Huang, Yuan Zhao, Chengju Liu, Qijun Chen} \\
  \footnotesize{\textit{ IEEE Transactions on Cognitive and Developmental Systems @ (TCDS).}}
   
\item \textbf{Sensor-Free Method with BP network to Achieve Drag Teaching on the 7-DOF Collaborative Robot} \quad [under review]\\
  \footnotesize{\textbf{Heng Zhang}, Xianyou Zhong, Zhengang Huang, Chengju Liu, Qijun Chen*} \\
  \footnotesize{\textit{ China Automation Conference 2021 @ (CAC2021).}}

\item \textbf{Control Method, Device and Equipment of Collaborative Robot Drag Teaching Based on Motor Current}\quad [Patents]\\
  \footnotesize{Qijun Chen, \textbf{Heng Zhang}, Chengju Liu}\\
  \footnotesize{\textit{ CN 112894821A.}}
  
\item \textbf{An indoor navigation method for Pepper robot}\quad [Patents]\\
  \footnotesize{Chengju Liu, Qijun Chen, Liwen Lu, Jiayuan Du, \textbf{Heng Zhang}}\\
  \footnotesize{\textit{ CN 113029143A.}} 
  
\item \textbf{Motion recognition of human skeleton based on lightweight graph convolution based on channel attention}\quad [Patents]\\
  \footnotesize{ Chengju Liu, Ronghao Dang, Qijun Chen, \textbf{Heng Zhang}}\\
  \footnotesize{\textit{ CN113111760A.}}
  
\item \textbf{A text feature construction method based on Word2Vec and syntactic dependency tree}\quad [Patents]\\
  \footnotesize{ Qijun Chen, Qiuchen Wang, Chengju Liu \textbf{Heng Zhang}}\\
  \footnotesize{\textit{ CN 113111653A.}}\\
       
\end{itemize}

%======================================================================
\sectionTitle{Honors \& Awards}{\faFlagCheckered}
%======================================================================
\begin{competences}[6em]
  \comptence{Scholarship}{
    \textbf{1.}Innovation Scholarship for Outstanding Students,
    \textbf{2.}Second Prize Scholarship}
  \comptence{Contests}{%
    \textbf{The First Prize} 2015 Siemens Cup National Industrial Automation Challenge Northeast Division,
    \textbf{The First prize} Intelligent robot in the 8th Science and Technology Sports Championship of Jilin Province,
    \textbf{The Third prize} National College Students Electronic Design Competition, JiLin
  }
  \comptence{Honors}{%
  \textbf{1.}Winner of The Excellent Graduate Papers,\textbf{2.}
    Outstanding student leader model
  }

\end{competences}




%======================================================================
\sectionTitle{Work \&Internships}{\faBriefcase}
%======================================================================
\begin{experiences}
  \experience%
    [Jul. 2020]%
    {Oct. 2020}%
    { Intern Algorithm Engineer @ China Railway Rolling stock Corp (CRRC), Tsingtao, China}%
    [\begin{itemize}
      \item Dynamics modeling on the 7-DOF collaborative robot.
      \item Drag teaching based on a sensor-free method
    \end{itemize}]\\

  \separator{0.5ex}
  \experience%
    [Sep. 2020]%
    {Apr. 2021}%
    {Participant Student @ Project: High performance universal robot control platform}%
    [\begin{itemize}
      \item Motion control of 6-DOF industrial robot in joint and Cartesian space.
      \item Communication configuration based on EthaerCAT.
    \end{itemize}]\\
    
    \separator{0.5ex}
  \experience%
    [Jul. 2016]%
    {Aug. 2019}%
    {Assistant Nuclear Power Plant Design Engineer @ SMNPC at China National Nuclear Corporation (CNNC)}%
    [\begin{itemize}
      \item Instrument and Control (I\&C) system design for AP1000 nuclear power plant.
      \item Learn design codes and specifications.
    \end{itemize}]
    
\end{experiences}


%======================================================================
\sectionTitle{Community Service}{\faChild}
%======================================================================
People are always in a community whenever and wherever, serving for community makes me have the spirit of dedication \& team-work, I am willing to make more contribution to the community.
 
\begin{competences}[6em]
  \comptence{TA}{TA in Linear System Theory and Design}
  \comptence{volunteer}{One-star volunteer of China Foundation for Poverty Alleviation}
  \comptence{journalist}{Student journalist in Alumni Association}
  \comptence{Monitor}{Serve as the monitor throughout my undergraduate years}

\end{competences}




\end{document}
